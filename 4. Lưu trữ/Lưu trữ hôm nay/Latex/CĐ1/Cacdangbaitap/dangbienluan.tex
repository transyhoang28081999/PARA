\subsection*{\textcolor{blue}{Dạng 4: Tìm điều kiện của tham số để phương trình thỏa mãn điều kiện cho trước}}

Tìm giá trị của tham số để hệ phương trình nhận $(x_0; y_0)$ là nghiệm.

Hệ phương trình
\[
\begin{cases}
a x + b y = c \\
a' x + b' y = c'
\end{cases}
\]
có nghiệm $(x_0; y_0)$ khi và chỉ khi
\[
\begin{cases}
a x_0 + b y_0 = c \\
a' x_0 + b' y_0 = c'
\end{cases}
\]

- Tìm giá trị của tham số để nghiệm của hệ thỏa mãn một số điều kiện khác.

\textbf{Bước 1.} Dựa vào điều kiện của nghiệm thiết lập phương trình có ẩn là tham số.

\textbf{Bước 2.} Giải phương trình tham số.

\textbf{Bước 3.} Kết luận.

%--- Ví dụ ---
\vspace{1.5em}
\textcolor{blue}{\textbf{Ví dụ:}} 
Cho hệ phương trình
\[
\begin{cases}
(m+1)x + n y = 3 \\
2m x + y = 2
\end{cases}
\]
Tìm $m, n$ để hệ phương trình có nghiệm $(x ; y) = (1 ; 2)$.

\textbf{\textit{Hướng dẫn giải}}

Hệ nhận $(x ; y) = (1 ; 2)$ là nghiệm nên:
\[
\begin{cases}
(m+1) \cdot 1 + n \cdot 2 = 3 \\
2m \cdot 1 + 2 = 2
\end{cases}
\]
Giải hệ:
\[
\begin{cases}
m + 2n = 2 \\
2m + 2 = 2
\end{cases}
\]
\[
\begin{cases}
m + 2n = 2 \\
2m = 0
\end{cases}
\]
\[
\begin{cases}
m + 2n = 2 \\
m = 0
\end{cases}
\]
\[
\begin{cases}
0 + 2n = 2 \\
m = 0
\end{cases}
\]
\[
\begin{cases}
n = 1 \\
m = 0
\end{cases}
\]

Vậy với 
\[
\begin{cases}
m = 0\\
n = 1
\end{cases}
\]
hệ phương trình nhận $(x ; y) = (1 ; 2)$ là nghiệm.

%--- Ví dụ mẫu ---
\vspace{1.5em}
\textcolor{blue}{\textbf{Ví dụ mẫu}}

\textbf{Ví dụ 1.} Cho hệ phương trình

\[
\begin{cases}
x + 2y = 3 \\
2x + y = m-2
\end{cases}
\]

Tìm $m$ để hệ phương trình có nghiệm duy nhất $(x_0; y_0)$ với $y_0 = x_0$.

\textbf{\textit{Hướng dẫn giải}}

Ta có
\[
\begin{cases}
x + 2y = 3 \\
2x + y = m-2
\end{cases}
\]
Từ phương trình trên: $x = 3 - 2y$\\
Thay vào phương trình dưới:
\begin{align*}
2(3 - 2y) + y &= m-2 \\
6 - 4y + y &= m-2 \\
6 - 3y &= m-2 \\
3y &= 8 - m \\
y &= \dfrac{8-m}{3}
\end{align*}

Khi đó:
\[
x = 3 - 2y = 3 - 2 \cdot \dfrac{8-m}{3} = \dfrac{9 - 16 + 2m}{3} = \dfrac{2m-7}{3}
\]
Vậy nghiệm của hệ là:
\[
\left(
\dfrac{2m-7}{3} ;
\dfrac{8-m}{3}
\right)
\]

Yêu cầu $y_0 = x_0$ nên:
\[
\dfrac{2m-7}{3} = \dfrac{8-m}{3}
\]
\begin{align*}
2m-7 &= 8-m \\
3m &= 15 \\
m &= 5
\end{align*}

Vậy với $m=5$, hệ phương trình có nghiệm duy nhất $(x_0 ; y_0)$ với $y_0 = x_0$.