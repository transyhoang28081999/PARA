\subsection*{\textcolor{blue}{Dạng 2: Giải hệ phương trình bằng phương pháp cộng đại số}}
Để giải một hệ hai phương trình bậc nhất hai ẩn \textit{có hệ số của cùng một ẩn nào đó trong hai phương trình bằng nhau hoặc đối nhau}, ta có thể làm như sau:\\[0.2cm]
\begin{itemize}[left=0pt, label={--}]
    \item \textbf{Bước 1.} Cộng hay trừ từng vế của hai phương trình trong hệ để được phương trình chỉ còn chứa một ẩn.
    \item \textbf{Bước 2.} Giải phương trình một ẩn vừa nhận được, từ đó suy ra nghiệm của hệ phương trình đã cho.
\end{itemize}
% Bài tập số 1:
\subsubsection*{Bài tập 1:}
Giải hệ phương trình sau:
$
\begin{cases}
2x+3y=7\\
4x-3y=1
\end{cases}
$
\\[0.3cm]
\textit{Lời giải:}\\[0.2cm]
Cộng từng vế của hai phương trình ta được: 
\[
6x=8 \quad\text{suy ra}\quad x=\frac{4}{3}.
\]
Thế $x=\frac{4}{3}$ vào phương trình đầu tiên ta được: 
\[
2.\frac{4}{3} + 3y = 7 \quad\text{suy ra}\quad y=\frac{13}{9}.
\]
Vậy hệ phương trình có nghiệm là \fbox{$(\frac{4}{3},\frac{13}{9})$}.
% Bài tập số 2:
\subsubsection*{Bài tập 2:}
Giải hệ phương trình sau:
$
\begin{cases}
x+2y=5\\
3x-y=4
\end{cases}
$
\\[0.3cm]
\textit{Lời giải:}\\[0.2cm]
Nhân cả hai vế của phương trình thứ hai với $2$ ta được: $6x-2y=8$.\\
Cộng từng vế của phương trình vừa tìm được với phương trình thứ nhất, ta được:
\[
7x=13
\]
Từ đây, ta suy ra: $x=\frac{13}{7}$.\\
Thay $x=\frac{13}{7}$ vào phương trình thứ 2, ta được:
\[
3.\frac{13}{7}-y=4 \quad\text{suy ra}\quad y=\frac{11}{7}.
\]
Vậy hệ phương trình có nghiệm là \fbox{$(\frac{13}{7},\frac{11}{7})$}.
% Bài tập số 3:
\subsubsection*{Bài tập 3:}
Giải hệ phương trình sau:
$
\begin{cases}
3x-2y=5 \\\
2x+y=8
\end{cases}
$
\\[0.3cm]
\textit{Lời giải:}\\[0.2cm]
Nhân cả hai vế của phương trình thứ hai với $2$ ta được:
\[
4x+2y=16
\]
Cộng từng vế của phương trình vừa tìm được với phương trình thứ nhất, ta được:
\[
7x=21 \quad\text{suy ra}\quad x=3.
\]
Thay $x=3$ vào phương trình thứ hai:
\[
2.3+y=8 \quad\text{suy ra}\quad y=2.
\]
Vậy hệ phương trình có nghiệm là \fbox{$(3,2)$}.