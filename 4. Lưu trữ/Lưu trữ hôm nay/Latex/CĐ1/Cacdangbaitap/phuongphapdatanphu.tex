\subsection*{\textcolor{blue}{Dạng 3: Giải hệ phương trình bằng phương pháp đặt ẩn phụ}}

\subsection*{\textcolor{blue}{Phương pháp giải}}

Thực hiện theo các bước sau:

\begin{itemize}[left=0pt, label={--}]
    \item \textbf{Bước 1.} Đặt điều kiện.
    \item \textbf{Bước 2.} Đặt ẩn phụ cho các biểu thức của hệ phương trình để đưa hệ về hệ phương trình bậc nhất hai ẩn. Chú ý điều kiện của ẩn phụ.
    \item \textbf{Bước 3.} Sử dụng phương pháp thế, cộng đại số để giải hệ phương trình theo ẩn phụ.
    \item \textbf{Bước 4.} Với các giá trị của ẩn phụ tìm được, thay vào biểu thức đặt ẩn phụ để xác định nghiệm của hệ phương trình.
    \item \textbf{Bước 5.} Kết luận.
\end{itemize}

\vspace{1em}
\textcolor{blue}{\textbf{Ví dụ:}} \textit{Giải hệ phương trình}
\[
\begin{cases}
\dfrac{1}{x} + \dfrac{2}{y} = 2 \\
\dfrac{3}{x} - \dfrac{4}{y} = -1
\end{cases}
\]

\textbf{Hướng dẫn giải:}

Điều kiện: $x \neq 0 ;\ y \neq 0$

Đặt $\dfrac{1}{x} = a ;\ \dfrac{1}{y} = b \quad (a, b \neq 0)$. Hệ phương trình đã cho trở thành:
\[
\begin{cases}
a + 2b = 2 \\
3a - 4b = 1
\end{cases}
\]

Giải hệ:
\[
\begin{cases}
a + 2b = 2 \\
3a - 4b = 1
\end{cases}
\]
Từ phương trình đầu, ta có $a = 2 - 2b$. Thay vào phương trình còn lại, ta được:
\begin{align*}
    3(2 - 2b) - 4b &= 1 \\
    6 - 6b - 4b &= 1 \\
    6 - 10b &= 1 \\
    -10b &= 1 - 6 \\
    -10b &= -5 \\
    b &= \dfrac{1}{2}
\end{align*}

Khi $b = \dfrac{1}{2}$ thì $a = 2 - 2 \times \dfrac{1}{2} = 2 - 1 = 1$.

Với $a = 1$ suy ra $\dfrac{1}{x} = 1 \Rightarrow x = 1$ (thỏa mãn).\\
Với $b = \dfrac{1}{2}$ suy ra $\dfrac{1}{y} = \dfrac{1}{2} \Rightarrow y = 2$ (thỏa mãn).

\textbf{Vậy nghiệm của hệ phương trình đã cho là $(x ; y) = (1 ; 2)$.}
% Ví dụ 1
\vspace{1em}
\textcolor{blue}{\textbf{Ví dụ 1:}} \textit{Giải hệ phương trình}
\[
\begin{cases}
\dfrac{3}{x-1} - \dfrac{4}{y+2} = -1 \\[1.2em]
\dfrac{1}{x-1} + \dfrac{2}{y+2} = \dfrac{4}{3}
\end{cases}
\]

\textbf{\textit{Hướng dẫn giải}}

Điều kiện: $x \neq 1;\; y \neq -2$

Đặt $ \dfrac{1}{x-1} = a;\; \dfrac{1}{y+2} = b\ (a, b \neq 0) $. Hệ phương trình đã cho trở thành:
\[
\begin{cases}
3a - 4b = -1 \\
a + 2b = \dfrac{4}{3}
\end{cases}
\]

Từ phương trình thứ hai, $a = \dfrac{4}{3} - 2b$. Thay vào phương trình đầu, ta có:
\begin{align*}
    &3\left(\dfrac{4}{3} - 2b\right) - 4b = -1 \\
    &4 - 6b - 4b = -1 \\
    &4 - 10b = -1 \\
    &-10b = -5 \\
    &b = \dfrac{1}{2}
\end{align*}
Khi $b = \dfrac{1}{2}$ thì $a = \dfrac{4}{3} - 2\times \dfrac{1}{2} = \dfrac{4}{3} - 1 = \dfrac{1}{3}$.
Với $a = \dfrac{1}{3}$ thì $\dfrac{1}{x-1} = \dfrac{1}{3}$, $x-1=3$, $x=4$ (thỏa mãn điều kiện).

Với $b = \dfrac{1}{2}$ thì $\dfrac{1}{y+2} = \dfrac{1}{2}$, $y+2=2$, $y=0$ (thỏa mãn điều kiện).

\textbf{Vậy nghiệm của hệ phương trình là } $(x;y) = (4;0)$.

% ===================
% Ví dụ 2
\vspace{1em}
\textcolor{blue}{\textbf{Ví dụ 2:}} \textit{Giải hệ phương trình}
\[
\begin{cases}
2\sqrt{x-2} - 3\sqrt{y+1} = -4 \\[1.2em]
3\sqrt{x-2} + 2\sqrt{y+1} = 7
\end{cases}
\]

\textbf{\textit{Hướng dẫn giải}}

Điều kiện: $x \geq 2;\; y \geq -1$

Đặt $a = \sqrt{x-2}$, $b = \sqrt{y+1}$ ($a \geq 0$, $b \geq 0$). Hệ phương trình trở thành:
\[
\begin{cases}
2a - 3b = -4 \\
3a + 2b = 7
\end{cases}
\]
Từ phương trình thứ hai, $b = -\dfrac{3}{2}a + \dfrac{7}{2}$. Thay vào phương trình đầu:
\begin{align*}
    &2a - 3\left(-\dfrac{3}{2}a + \dfrac{7}{2}\right) = -4 \\
    &2a + \dfrac{9}{2}a - \dfrac{21}{2} = -4 \\
    &\dfrac{13}{2}a - \dfrac{21}{2} = -4 \\
    &\dfrac{13}{2}a = -4 + \dfrac{21}{2} \\
    &\dfrac{13}{2}a = \dfrac{13}{2} \\
    &a = 1
\end{align*}
Khi $a = 1$ thì $b = -\frac{3}{2} \cdot 1 + \frac{7}{2} = 2$.

Với $a = 1$ thì $\sqrt{x-2} = 1$ nên $x-2 = 1$, $x=3$ (thỏa mãn điều kiện).

Với $b=2$ thì $\sqrt{y+1}=2$ nên $y+1=4$, $y=3$ (thỏa mãn điều kiện).

\textbf{Vậy hệ phương trình có nghiệm duy nhất $(x;y) = (3;3)$.}