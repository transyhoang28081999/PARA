% \documentclass[12pt,a4paper]{article}
% \usepackage[utf8]{inputenc}
% \usepackage[vietnamese]{babel}
% \usepackage{amsmath}
% \usepackage{graphicx}
% \usepackage{geometry}
% \usepackage{xcolor}
% \usepackage{multicol}
% \usepackage{amsmath,amsxtra,amssymb,latexsym, amscd,amsthm}


% \begin{document}

\section*{Bài 1: Giải các hệ phương trình sau}

\begin{multicols}{3}
\begin{enumerate}
    \item $\begin{cases}
        x - 2y = -4 \\
        2x - y = 7
    \end{cases}$

    \item $\begin{cases}
        x - 3y = 8 \\
        2x - y = -5
    \end{cases}$

    \item $\begin{cases}
        x - y = 12 \\
        x + y = 6
    \end{cases}$

    \item $\begin{cases}
        3x - y = 7 \\
        5x + 2y = 11
    \end{cases}$

    \item $\begin{cases}
        \dfrac{1}{2}x + y = 3 \\
        3x + 2y = 8
    \end{cases}$

    \item $\begin{cases}
        \dfrac{y}{4} - \dfrac{x-y}{3} = \dfrac{1}{6} \\
        \dfrac{y}{3} - \dfrac{x+y}{4} = \dfrac{1}{4}
    \end{cases}$

    \item $\begin{cases}
        \dfrac{x}{3} - \dfrac{y}{4} = 0 \\
        \dfrac{5}{y+2} = \dfrac{7}{x+6}
    \end{cases}$

    \item $\begin{cases}
        x - y = 15 \\
        x - \dfrac{x}{6} = y + \dfrac{x}{10}
    \end{cases}$

    \item $\begin{cases}
        x - 2\sqrt{2}y = \sqrt{5} \\
        \sqrt{3}x + y = 2 - \sqrt{2}
    \end{cases}$

    \item $\begin{cases}
        x - y\sqrt{2} = 1 \\
        x\sqrt{2} + 2y = 2 + \sqrt{3}
    \end{cases}$

    \item $\begin{cases}
        \sqrt{3}x - \sqrt{2}y = 2 \\
        x + \sqrt{2}y = \sqrt{5}
    \end{cases}$

    \item $\begin{cases}
        \sqrt{2}x + \sqrt{7}y = 3 \\
        x + \sqrt{7}y = 1
    \end{cases}$
\end{enumerate}
\end{multicols}
\bigskip
\textit{\textbf{CHÚ Ý:}} Giải các bài toán trên bằng cách sử dụng phương pháp thế. 
\bigskip

 \section*{Bài 2: Giải các hệ phương trình sau}
\begin{multicols}{2}
\begin{enumerate}
    \item $\begin{cases}
        4x - 3y + 5(x-y) = 1 \\
        2x - 4(2y-1) = 1
    \end{cases}$

    \item $\begin{cases}
        2(x+1) - 15(y-1) = 8 \\
        3(x+1) - 2(y-1) = 1
    \end{cases}$

    \item $\begin{cases}
        5(x-y) - 3(2x+3y) = 12 \\
        3(x+2y) - 4(x+2y) = 5
    \end{cases}$

    \item $\begin{cases}
        \dfrac{2x+3}{y-1} = \dfrac{4x+1}{2y+1} \\
        \dfrac{x+2}{y-1} = \dfrac{x-4}{y+2}
    \end{cases}$

    \item $\begin{cases}
        \dfrac{x-y}{2} + \dfrac{x-3y}{4} = 0 \\
        \dfrac{3x-5y+1}{2} - 1 = 0
    \end{cases}$

    \item $\begin{cases}
        \tfrac{1}{2}(x+2)(y+3) = \tfrac{1}{2}xy + 50 \\
        \tfrac{1}{2}(x-2)(y-2) = \tfrac{1}{2}xy - 32
    \end{cases}$

    \item $\begin{cases}
        (x+2)(y-2) = xy \\
        (x+4)(y-3) = xy+6
    \end{cases}$

    \item $\begin{cases}
        (x-1)(y-2) - (x+1)(y-3) = 4 \\
        (x-3)(y+1) - (x-5)(y-5) = 18
    \end{cases}$

    \item $\begin{cases}
        (x+5)(y-2) = xy \\
        (x-5)(y+12) = xy
    \end{cases}$

    \item $\begin{cases}
        (x-1)(y-2) = (x+1)(y-3) \\
        (x-5)(y+4) = (x-4)(y+1)
    \end{cases}$

    \item $\begin{cases}
        (3x-7) - 6(x+y+1) = 0 \\
        4(x+1) - 2(x+2y+7) = 0
    \end{cases}$

    \item $\begin{cases}
        5(x+2y) - 3(x-y) = 99 \\
        x - 3y = 7x - 4y - 17
    \end{cases}$

    \item $\begin{cases}
        3(y-5) + 2(2-3x) = 0 \\
        7(x-4) + 3(x+y-1) = 14
    \end{cases}$

    \item $\begin{cases}
        2(x+1) - 5(y+1) = 8 \\
        3(x+1) - 2(y+1) = 1
    \end{cases}$

    \item $\begin{cases}
        2(3y+1) - 4(x-1) = 5 \\
        5(3y+1) - 8(x-1) = 9
    \end{cases}$

    \item $\begin{cases}
        3(x+y) - 2(x-y) = 9 \\
        2(x+y) + (x-y) = -1
    \end{cases}$

    \item $\begin{cases}
        x^2 + 3y = 1 \\
        3x^2 - y = 1
    \end{cases}$

    \item $\begin{cases}
        (x-3)(2y+5) = (2x+7)(y-1) \\
        4(x+3y-6) = (6x-2)(y+3)
    \end{cases}$

    \item $\begin{cases}
        2(2x+3y) = 3(2x-3y) + 10 \\
        4x - 3y = 4(6y-2x) + 3
    \end{cases}$

    \item $\begin{cases}
        -x + 2y = -4(x-1) \\
        5x + 3y = (x+y) + 8
    \end{cases}$

    \item $\begin{cases}
        (\sqrt{3}-\sqrt{2})x = y\sqrt{2} \\
        x + (\sqrt{3}+\sqrt{2})y = \sqrt{6}
    \end{cases}$

    \item $\begin{cases}
        (x-2)+3(1+y) = -2 \\
        3(x-2) - 2(1+y) = -3
    \end{cases}$

    \item $\begin{cases}
        2(x+y) + 3(x-y) = 4 \\
        (x+y) + 2(x-y) = 5
    \end{cases}$

    \item $\begin{cases}
        (3x+1) + 2y = -x \\
        5(x+y) = -3x + y - 5
    \end{cases}$
\end{enumerate}
\end{multicols}
\bigskip
\textit{\textbf{CHÚ Ý:}} Rút gọn từng phương trình của hệ sau đó giải hệ bằng phương pháp thế hoặc cộng đại số. 
\bigskip



\section*{Bài 3: Giải các hệ phương trình sau}
\begin{multicols}{2}
\begin{enumerate}
    \item $\begin{cases}
        \dfrac{2}{x+y} + \dfrac{1}{x-y} = 3 \\
        \dfrac{1}{x+y} - \dfrac{3}{x-y} = 1
    \end{cases}$

    \item $\begin{cases}
        \dfrac{1}{x-2} + \dfrac{1}{y-1} = 2 \\
        \dfrac{2}{x-2} - \dfrac{3}{y-1} = 1
    \end{cases}$

    \item $\begin{cases}
        \dfrac{2}{3x-y} - \dfrac{5}{x-3y} = 3 \\
        \dfrac{1}{3x-y} + \dfrac{2}{x-3y} = \dfrac{3}{5}
    \end{cases}$

    \item $\begin{cases}
        \dfrac{1}{x+y} - \dfrac{2}{x-y} = 2 \\
        \dfrac{5}{x+y} - \dfrac{4}{x-y} = 3
    \end{cases}$

    \item $\begin{cases}
        \dfrac{3}{2x-y} + \dfrac{5}{2x+y} = 2 \\
        \dfrac{1}{2x-y} + \dfrac{1}{2x+y} = \dfrac{2}{15}
    \end{cases}$

    \item $\begin{cases}
        \dfrac{2}{x} + \dfrac{5}{x+y} = 2 \\
        \dfrac{3}{x} + \dfrac{1}{x+y} = 1.7
    \end{cases}$

    \item $\begin{cases}
        \dfrac{5}{x-2} + \dfrac{3}{y-1} = 2 \\
        \dfrac{2}{x-2} - \dfrac{5}{y-1} = 1
    \end{cases}$

    \item $\begin{cases}
        \dfrac{4}{2x+1} + \dfrac{9}{y-1} = 3 \\
        \dfrac{3}{2x+1} - \dfrac{2}{y-1} = \dfrac{13}{6}
    \end{cases}$

    \item $\begin{cases}
        \dfrac{4}{2x-3y} + \dfrac{5}{3x+y} = -2 \\
        \dfrac{3}{3x+y} - \dfrac{5}{2x-3y} = 21
    \end{cases}$

    \item $\begin{cases}
        \dfrac{6x-3}{y-1} - \dfrac{2y}{x+1} = 5 \\
        \dfrac{4x-2}{y-1} + \dfrac{y}{x+1} = 3
    \end{cases}$

    \item $\begin{cases}
        \dfrac{5}{x+y-3} - \dfrac{2}{x-y+1} = 8 \\
        \dfrac{3}{x+y-3} + \dfrac{1}{x-y+1} = 1.5
    \end{cases}$

    \item $\begin{cases}
        \dfrac{4}{x+y-1} + \dfrac{5}{2x-y+3} = 2 \\
        \dfrac{3}{x+y-1} - \dfrac{7}{2x-y+3} = \dfrac{7}{5}
    \end{cases}$

    \item $\begin{cases}
        \dfrac{x}{y} - \dfrac{x}{y+12} = 1 \\
        \dfrac{x}{y+12} + \dfrac{x}{y} = 2
    \end{cases}$

    \item $\begin{cases}
        \dfrac{5x}{y-1} - \dfrac{x}{y-3} = 27 \\
        \dfrac{2x}{y-1} - \dfrac{3y}{y-3} = 4
    \end{cases}$

    \item $\begin{cases}
        \dfrac{2x}{y-1} - \dfrac{3y}{x-1} = 1 \\
        \dfrac{x-1}{y-1} - \dfrac{x}{y-1} = 2
    \end{cases}$

    \item $\begin{cases}
        2x^2 + 3y^2 = 36 \\
        3x^2 + 7y^2 = 37
    \end{cases}$

    \item $\begin{cases}
        3x^2 + y^2 = 5 \\
        x^2 - 3y^2 = 1
    \end{cases}$

    \item $\begin{cases}
        4x^2 + y^2 = 13 \\
        2x^2 - y^2 = -7
    \end{cases}$
\end{enumerate}
\end{multicols}
\bigskip

\section*{Bài 4: Cho hệ phương trình}
$\begin{cases}
x - 3y = 7 \quad (1) \\
mx - 2y = 5 \quad (2)
\end{cases}$
\begin{enumerate}
    \item[a)] Giải hệ phương trình với $m = 3$.
    \item[b)] Tìm $m$ để hệ phương trình có nghiệm duy nhất $(x,y)$ trong đó $x,y$ trái dấu.
    \item[c)] Tìm $m$ để hệ phương trình có nghiệm duy nhất $(x,y)$ thỏa mãn $x = |y|$.
\end{enumerate}
\bigskip

\section*{Bài 5: Cho hệ phương trình}
$\begin{cases}
x + m y = 2m - 3\\
m x + y = m + 4
\end{cases}$
\qquad (\,m \text{ là tham số}\,)

\begin{enumerate}
    \item[a)] Giải hệ phương trình khi $m = 3$.
    \item[b)] Tìm $m$ để hệ phương trình có nghiệm duy nhất $(x,y)$ thỏa mãn
    \(
    \left\{
    \begin{aligned}
        &x \ge 1\\
        &y \ge 2
    \end{aligned}
    \right.
    \).
\end{enumerate}
\bigskip

\section*{Bài 6: Xác định giá trị của các hệ số $m, n$ sao cho:}
\begin{enumerate}
  \item[a.] Hệ phương trình
  $\begin{cases}
      3x + my = n \\
      mx + 2ny = 7
   \end{cases}$
   có nghiệm là $x = 1;\ y = 3$?

  \item[b.] Hệ phương trình
  $\begin{cases}
      x - 2y = m \\
      4x + 3y - n = 2
   \end{cases}$
   có nghiệm là $x = 2;\ y = -1$?
\end{enumerate}
\bigskip

\section*{Bài 7: Giải các hệ phương trình sau}

\begin{enumerate}
    \item[a)] ĐKXĐ: $x+y\ne 0,\ x\ne 1$.
    \[
    \begin{cases}
    \dfrac{2}{x+y} + \dfrac{1}{x-1} = 3,\\[4pt]
    \dfrac{1}{x+y} - \dfrac{3}{x-1} = 1.
    \end{cases}
    \]

    \item[b)] ĐKXĐ: $x\ge -2,\ y\ge 1$.
    \[
    \begin{cases}
    \sqrt{x+2}+\sqrt{\,y-1\,}=5,\\
    x-y=3.
    \end{cases}
    \]

    \item[c)] ĐKXĐ: $x\ne 1,\ y\ne -2$.
    \[
    \begin{cases}
    \dfrac{x}{x-1}+\dfrac{y}{y+2}=3,\\[6pt]
    \dfrac{x}{x-1}-\dfrac{y}{y+2}=2.
    \end{cases}
    \]
\end{enumerate}
\bigskip

\section*{Bài 8: Biện luận số nghiệm của hệ phương trình}

\begin{enumerate}
    \item Cho hệ:
    \[
    \begin{cases}
    (m-1)x + y = 2,\\
    2x + (m+1)y = 3.
    \end{cases}
    \]
    Hãy biện luận theo $m$ số nghiệm của hệ.

    \item Cho hệ:
    \[
    \begin{cases}
    x + my = 1,\\
    mx + y = m.
    \end{cases}
    \]
    Tìm $m$ để hệ có nghiệm duy nhất, vô nghiệm, vô số nghiệm.

    \item Cho hệ:
    \[
    \begin{cases}
    (m+2)x - y = 1,\\
    2x + (m-3)y = 5.
    \end{cases}
    \]
    Biện luận theo $m$ số nghiệm của hệ.

    \item Cho hệ:
    \[
    \begin{cases}
    mx - y = 2,\\
    2x + (m-1)y = m.
    \end{cases}
    \]
    Biện luận số nghiệm theo $m$.

    \item Cho hệ:
    \[
    \begin{cases}
    (m+1)x + (m-1)y = 3,\\
    (m-1)x + (m+1)y = 1.
    \end{cases}
    \]
    Biện luận số nghiệm của hệ.

    \item Cho hệ:
    \[
    \begin{cases}
    (m-2)x + 3y = 5,\\
    2x + (m+4)y = m.
    \end{cases}
    \]
    Tìm các giá trị của $m$ để hệ có nghiệm duy nhất, vô nghiệm, vô số nghiệm.

    \item Cho hệ:
    \[
    \begin{cases}
    (m+1)x - y = m,\\
    (m-1)x + (m+2)y = 2.
    \end{cases}
    \]
    Biện luận số nghiệm theo $m$.

    \item Cho hệ:
    \[
    \begin{cases}
    mx + y = 1,\\
    (m-1)x + (m+1)y = 2.
    \end{cases}
    \]
    Biện luận số nghiệm theo $m$.

    \item Cho hệ:
    \[
    \begin{cases}
    (m+2)x + (m-1)y = 0,\\
    (m-1)x + (m+2)y = 0.
    \end{cases}
    \]
    Biện luận số nghiệm theo $m$.

    \item Cho hệ:
    \[
    \begin{cases}
    (m-3)x + 2y = 1,\\
    4x + (m-3)y = 2.
    \end{cases}
    \]
    Biện luận số nghiệm theo $m$.
\end{enumerate}

% \end{document}

